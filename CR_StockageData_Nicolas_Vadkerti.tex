\documentclass[a4paper,10pt]{article}
\usepackage[utf8]{inputenc}
\usepackage[french]{babel}
\usepackage{hyperref}
%opening
\title{Stockage de Données avec base de données MySQLd}
\author{Nicolas Vadkerti}
\usepackage{listings} % Required for inserting code snippets
\usepackage[usenames,dvipsnames]{color} % Required for specifying custom colors and referring to colors by name
\usepackage{graphicx}
\usepackage[left=2cm,right=2cm,top=2cm,bottom=2cm]{geometry}



\definecolor{DarkGreen}{rgb}{0.0,0.4,0.0} % Comment color
\definecolor{highlight}{RGB}{255,251,204} % Code highlight color


\lstdefinestyle{Style1}{ % Define a style for your code snippet, multiple definitions can be made if, for example, you wish to insert multiple code snippets using different programming languages into one document
language=Perl, % Detects keywords, comments, strings, functions, etc for the language specified
backgroundcolor=\color{highlight}, % Set the background color for the snippet - useful for highlighting
basicstyle=\footnotesize\ttfamily, % The default font size and style of the code
breakatwhitespace=false, % If true, only allows line breaks at white space
breaklines=true, % Automatic line breaking (prevents code from protruding outside the box)
captionpos=b, % Sets the caption position: b for bottom; t for top
commentstyle=\usefont{T1}{pcr}{m}{sl}\color{DarkGreen}, % Style of comments within the code - dark green courier font
deletekeywords={}, % If you want to delete any keywords from the current language separate them by commas
%escapeinside={\%}, % This allows you to escape to LaTeX using the character in the bracket
firstnumber=1, % Line numbers begin at line 1
frame=single, % Frame around the code box, value can be: none, leftline, topline, bottomline, lines, single, shadowbox
frameround=tttt, % Rounds the corners of the frame for the top left, top right, bottom left and bottom right positions
keywordstyle=\color{Blue}\bf, % Functions are bold and blue
morekeywords={}, % Add any functions no included by default here separated by commas
numbers=left, % Location of line numbers, can take the values of: none, left, right
numbersep=10pt, % Distance of line numbers from the code box
numberstyle=\tiny\color{Gray}, % Style used for line numbers
rulecolor=\color{black}, % Frame border color
showstringspaces=false, % Don't put marks in string spaces
showtabs=false, % Display tabs in the code as lines
stepnumber=5, % The step distance between line numbers, i.e. how often will lines be numbered
stringstyle=\color{Purple}, % Strings are purple
tabsize=2
}

\newcommand{\insertcode}[2]{\begin{itemize}\item[]\lstinputlisting[caption=#2,label=#1,style=Style1]{#1}\end{itemize}} 


% \insertcode{"Scripts/example.pl"}{Nena would be proud.} 

\begin{document}

\maketitle


\url{https://github.com/SlaynPool/CR_BDD/}

\section{Préparation (Installation des outils)}

 Suite  au passage à Debian 10, PHPmyAdmin n'est plus disponnible via les depots de APT, de ce fais, j'ai décidé d'utiliser Adminer, un gestionnaire de DATABASE equivalent, qui a la force d'etre qu'un seul fichier PHP donc très simple de déploiment.
 
 \insertcode{commande/1.txt}{Installation de Adminer }
\newpage
 \section{Mise en jambes}
 \subsection{Analyse de la base de test}
 \subsubsection{Analyser la nature de la base. Combien y a-t-il de tables dans la base?}
 La base a pour but de stocker des informations sur des pays, On peut voir qu'il ya trois tables dans la base :
 \insertcode{commande/2.txt}{SHOW TABLES}
 
  \subsubsection{Quelles sont les clefs primaire et secondaire des tables}
 Si l'on prend l'exemple de la table city :
 \insertcode{commande/3.txt}{SELECT * FROM city;}
 La pkey est  ID , et les clés secondaires sont : Name, CountryCode, District, Population.
 Cependant, les autres tables ont pour pkey : Code pour country et  CountryCode et language pour countrylanguage
 
 
 \subsection{Requetes de Base}
 \subsubsection{Donner la liste de toutes les villes françaises}
  \begin{figure}[h!]
\centering
\includegraphics[scale=0.8]{ressource/1.jpg}
\caption{Notre requete SQL`}
\label{fig:paquet}
\end{figure}
\subsubsection{Récupérer la liste de toutes les capitales.}
Pour cette question j'ai trouvé deux solutions :

  \begin{figure}[h!]
\centering
\includegraphics[scale=0.8]{ressource/2.jpg}
\caption{Solution 2}
\label{fig:paquet}
\end{figure}

\includegraphics[scale=0.8]{ressource/3.jpg}

\newpage


\subsubsection{Combien y a-t-il de pays dans la base ? }
\insertcode{commande/5.txt}{count() }
\subsubsection{Donner les villes dont la population est supérieure à 1 million d’habitants. Combien de villescorrespondent à ce critère?}

\insertcode{commande/6.txt}{Requetes}

\subsubsection{Faire un script php permettant de générer une page web affichant le résultat d’une requête sql. Tester le script avec les requêtes des questions précédentes. Le template d’une page webest donné ci-dessous.}
Pour cette question, j'ai donc ecrit un script qui a pour role d'afficher sur une page web le resultat des requètes SQL des questions précedentes. Voici le résultat :
\insertcode{ressource/code.php}{php script}

\section{Stockage de données pour l’IDO}
w

\end{document}

